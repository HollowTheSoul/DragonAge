\documentclass[12,english]{article}
\usepackage[letterpaper, portrait, margin=1in]{geometry}

\usepackage{amsmath}
\usepackage[T1]{fontenc}
\usepackage{babel}
\usepackage{textcomp}
\usepackage{titlesec}
\setcounter{secnumdepth}{4}
\usepackage{hyperref}
\hypersetup{
    bookmarks=true,         % show bookmarks bar?
      colorlinks=true,       % false: boxed links; true: colored links
    linkcolor=black,          % color of internal links (change box color with linkbordercolor)
    citecolor=green,        % color of links to bibliography
    filecolor=magenta,      % color of file links
    urlcolor=cyan           % color of external links
}




\title{SE 3XA3: Module Interface Specification\\
        Dragon Age}
\author{Group 8: Team Eight \\
                 Stanley Liu (MacID: liuz23) \\    
                 Toni Miharja (MacID: miharjat)\\
                 Zhi Zhang (MacID: zhangz1)}
\date{November 10 2017 }
\usepackage[showgrame]{geometry}
\usepackage{titling}
\renewcommand\maketitlehooka{\null\mbox{}\vfill}
\renewcommand\maketitlehookd{\vfill\null}
\setcounter{secnumdepth}{4}

\begin{document}
\maketitle
\newpage
\tableofcontents
\newpage
 
\section{Revision History}
\begin{table}[h!]
    \centering
    \begin{tabular}{|p{2.5cm}|p{3cm}|p{3cm}|p{2cm}|}
    \hline
    \textbf {Date}  & {Developer} & {Change} & {Revision} \\
    \hline
    November 10, 2017 & Zhi & Part 3, 5, 7, 8 & 1.0\\
    \hline
    November 10, 2017 & Stanley & Part 2, 4, 6 & 1.1\\
    \hline
    November 10, 2017 & Toni & Part 9, 10, 11, 12, 13, 14& 1.2\\
    \hline
    \end{tabular}
    \caption{Revision History: Module Interface Specification}
\end{table}
 
\section{Module Hierarchy}
\begin{table}[h!]
    \centering
    \begin{tabular}{|p{5.5cm}|p{5cm}|}
    \hline
    \textbf {Level 1}  & {Level 2} \\
    \hline
    Hardware Hiding Module &  \\
    \hline
    Behaviour Hiding Module & Dragon Tower Module\\
    & Time Bullet Module\\
    &Time Enemy Module\\
    &Time Hover Module\\
    &Time Fired Module\\
    &Draw Module\\
    &Game Manager Module\\
    &Gragon Age Module\\
    \hline
    Software Decision Hiding Module&Dragon Module\\
    & Enemy Module\\
    & Bullet Module\\
    & Path Module\\
    & Game Date Module\\
    \hline
    \end{tabular}
    \caption{Revision History: Module Hierarchy}
\end{table}

\section{MIS of Dragon Tower Module}
\subsection{Interface Syntax}
\subsubsection{Exported Access Programs}
\begin{table}[h!]
    \centering
    \begin{tabular}{|p{4cm}|p{2cm}|p{2cm}|p{2cm}|}
    \hline
    \textbf {Name}  & {In} & {Out} & {Exceptions} \\
    \hline
    setDragons & - & - & -\\
    \hline
    isInRangeEquation & - & float & -\\
    \hline
    isInRange & - & boolean & -\\
    \hline
    drawTower & - & - & -\\
    \hline
    drawRadius & - & - & Insufficient building space\\
    \hline
    canEvolve & - & boolean & No tower is been built\\
    \hline
    evolve & - & - & Highest level reached\\
    \hline
    \end{tabular}
\end{table}
\newpage

\subsection{Interface Semantics}
\subsubsection{State Variables}
Not Applicable
\subsubsection{Environmental Variables}
Not Applicable
\subsubsection{Assumptions}
Game started. 
\subsubsection{Access Program Semantics}
\noindent setDragons():
\begin{itemize}
    \item Output : set three types of dragons
    \item Exceptions: None
\end{itemize}

\noindent isInRangeEquation(x,y):
\begin{itemize}
    \item Input: x, y
    \item Transition: get inRange value from x, y
    \item Output: inRange
\end{itemize}

\noindent isInRange(bounds):
\begin{itemize}
    \item Transition: x0, x1, y0, y1 := bounds
    \item Output: return the boolean value whether or not the enemy is in range of tower
\end{itemize}

\noindent drawTower(canvas):
\begin{itemize}
    \item Exceptions: insufficient building space
\end{itemize}

\noindent drawRadius(canvas):
\begin{itemize}
    \item Output: draw radis of the enemy
\end{itemize}

\noindent canEvolve(canvas):
\begin{itemize}
    \item Output:  whether or not the tower can still evolve
    \item Exceptions: No tower is been built
\end{itemize}

\noindent evolve():
\begin{itemize}
    \item Transition: dragon tower evolve to its next level
    \item Exception:  Highest level reached
\end{itemize}

\section{MIS of Timer Bullet Module}
\subsection{Interface Syntax}
\subsubsection{Exported Access Programs}
\begin{table}[h!]
    \centering
    \begin{tabular}{|p{4cm}|p{2cm}|p{2cm}|p{2cm}|}
    \hline
    \textbf {Name}  & {In} & {Out} & {Exceptions} \\
    \hline
    moveAllBullets & - & - & Bullet out of bound\\
    \hline
    removeBullets & - & - & -\\
    \hline 
    setTarget & - & - & -\\
    \hline
    shootEnemies & - & - & -\\
    \hline
    setDamage & float & float & -\\
    \hline
    setBullets & - & - & -\\
    \hline
    allBulletsRemoved & - & boolean & -\\
    \hline
    \end{tabular}
\end{table}
\subsection{Interface Semantics}
\subsubsection{State Variables}
Not Applicable
\subsubsection{Environmental Variables}
Not Applicable
\subsubsection{Assumptions}
Assume dragon tower is placed onto board
\subsubsection{Access Program Semantics}

\noindent moveAllBullets():
\begin{itemize}
    \item Transition: move bullets, if bullet goes out of bounds, remove bullets
    \item Exception: bullet out of bound
\end{itemize}

\noindent removeBullets():
\begin{itemize}
    \item Input: x, y
    \item Transition: check whether bullets are removed for every frame and replace bullet list
\end{itemize}

\noindent setTarget():
\begin{itemize}
    \item Transition: set target for each tower
\end{itemize}

\noindent shootEnemies():
\begin{itemize}
    \item Transition: check if bullet hits enemy, if hit, set damage done to enemy or if enemy dies, enemy exit board and player get coins
\end{itemize}

\noindent setDamage():
\begin{itemize}
    \item Output: the damage done to enemy
\end{itemize}

\noindent setBullets():
\begin{itemize}
    \item Transition: set bullets for tower if tower has a target
\end{itemize}

\noindent allBulletRemoved():
\begin{itemize}
    \item Transition: check if all bullets are removed from the board
    \item Output: return the value of bullet.remove
\end{itemize}


\section{MIS of Timer Enemy Module}
\subsection{Interface Syntax}
\subsubsection{Exported Access Programs}
\begin{table}[h!]
    \centering
    \begin{tabular}{|p{4cm}|p{2cm}|p{2cm}|p{2cm}|}
    \hline
    \textbf {Name}  & {In} & {Out} & {Exceptions} \\
    \hline
    moveAllEnemies & - & boolean & -\\
    \hline
    roundOver & - & - & - \\
    \hline 
    \end{tabular}
\end{table}

\subsection{Interface Semantics}
\subsubsection{State Variables}
None
\subsubsection{Environmental Variables}
Not Applicable
\subsubsection{Assumptions}
Enemy are running on the board. 
\subsubsection{Access Program Semantics}

\noindent moveAllEnemies():
\begin{itemize}
    \item Output : move enemies on the board at different speed
    \item Exceptions: None
\end{itemize}

\noindent roundOver(x,y):
\begin{itemize}
    \item Output: whether or not the round is over
    \item Exception: None
\end{itemize}

\newpage
\section{MIS of Timer Hover Module}
\subsection{Interface Syntax}
\subsubsection{Exported Access Programs}
\begin{table}[h!]
    \centering
    \begin{tabular}{|p{4cm}|p{2cm}|p{2cm}|p{2cm}|}
    \hline
    \textbf {Name}  & {In} & {Out} & {Exceptions} \\
    \hline
    hover & - & - & -\\
    \hline
    buildTowerHover& real, real & - & - \\
    \hline 
    \end{tabular}
\end{table}

\subsection{Interface Semantics}
\subsubsection{State Variables}
Not Applicable
\subsubsection{Environmental Variables}
None
\subsubsection{Assumptions}
The game is started. 
\subsubsection{Access Program Semantics}

\noindent hover():
\begin{itemize}
    \item Transition: x,y := pygame.mouse.get\_pos()
    \item Output: put the tower on board
\end{itemize}

\noindent buildTowerHover(x,y):
\begin{itemize}
    \item Input: x, y cordinates
    \item Transition: gameData.playerSelected.x, gameData.playerSelected.y= x,y
    \item Output: draw rectangle of size of dragon when building is legal
\end{itemize}

\section{MIS of Timer Fired Module}
\subsection{Interface Syntax}
\subsubsection{Exported Access Programs}
\begin{table}[h!]
    \centering
    \begin{tabular}{|p{4cm}|p{2cm}|p{2cm}|p{2cm}|}
    \hline
    \textbf {Name}  & {In} & {Out} & {Exceptions} \\
    \hline
    timeFired & - & - & -\\
    \hline 
    \end{tabular}
\end{table}
\subsection{Interface Semantics}
\subsubsection{State Variables}
Not Applicable 
\subsubsection{Environmental Variables}
None
\subsubsection{Assumptions}
None
\subsubsection{Access Program Semantics}

\noindent timeFired():
\begin{itemize}
    \item Transition: runs all the time-based modules of the game
\end{itemize}

\section{MIS of Draw Module}
\subsection{Interface Syntax}
\subsubsection{Exported Access Programs}
\begin{table}[h!]
    \centering
    \begin{tabular}{|p{4cm}|p{2cm}|p{2cm}|p{2cm}|}
    \hline
    \textbf {Name}  & {In} & {Out} & {Exceptions} \\
    \hline
    drawIntro & - & - & -\\
    \hline 
    drawEnemies & - & - & -\\
    \hline 
    drawPlay & - & - & -\\
    \hline 
    drawTowers & - & - & -\\
    \hline 
    drawParty & - & - & -\\
    \hline 
    drawAllBullets & - & - & -\\
    \hline 
    drawAll & - & - & -\\
    \hline 
    \end{tabular}
\end{table}
\subsection{Interface Semantics}
\subsubsection{State Variables}
Not Applicable 
\subsubsection{Environmental Variables}
None
\subsubsection{Assumptions}
The game is started. 
\subsubsection{Access Program Semantics}

\noindent drawIntro():
\begin{itemize}
    \item Transition: display the introduction page on the board
\end{itemize}

\noindent drawEnemies():
\begin{itemize}
    \item Transition: display enemies on the board
\end{itemize}

\noindent drawPlay():
\begin{itemize}
    \item Transition: display Play Button
\end{itemize}

\noindent drawTowers():
\begin{itemize}
    \item Transition: draw all towers on board
\end{itemize}

\noindent drawParty():
\begin{itemize}
    \item Transition: display options of dragon towers for game player 
\end{itemize}

\noindent drawAllBullets():
\begin{itemize}
    \item Transition: draw all bullets on board
\end{itemize}

\noindent drawAll():
\begin{itemize}
    \item Transition: draw all items above on the board
\end{itemize}


\section{MIS of Game Manager Module}
\subsection{Interface Syntax}
\subsubsection{Exported Access Programs}
\begin{table}[h!]
    \centering
    \begin{tabular}{|p{4cm}|p{2cm}|p{2cm}|p{2cm}|}
    \hline
    \textbf {Name}  & {In} & {Out} & {Exceptions} \\
    \hline
    gameInit & - & - & -\\
    \hline 
    runGame & - & - & -\\
    \hline 
    mousePress & int, int & - & -\\
    \hline 
    \end{tabular}
\end{table}
\subsection{Interface Semantics}
\subsubsection{State Variables}
Not Applicable 
\subsubsection{Environmental Variables}
None
\subsubsection{Assumptions}
None
\subsubsection{Access Program Semantics}

\noindent gameInit():
\begin{itemize}
    \item Transition: Initialise the game data
\end{itemize}

\noindent runGame():
\begin{itemize}
    \item Transition: The functions that will be run continuously in the while loop of the main game
\end{itemize}

\noindent mousePress(x,y):
\begin{itemize}
    \item Input: x and y coordinates
    \item Transition: Handle the mouse control of the game
\end{itemize}

\section{MIS of Dragon Age Module}
\subsection{Interface Syntax}
\subsubsection{Exported Access Programs}
\begin{table}[h!]
    \centering
    \begin{tabular}{|p{4cm}|p{2cm}|p{2cm}|p{2cm}|}
    \hline
    \textbf {Name}  & {In} & {Out} & {Exceptions} \\
    \hline
    init & - & - & -\\
    \hline 
    mouse & - & - & -\\
    \hline 
    loadBackground & - & - & -\\
    \hline 
    game & - & - & -\\
    \hline 
    \end{tabular}
\end{table}
\subsection{Interface Semantics}
\subsubsection{State Variables}
Not Applicable 
\subsubsection{Environmental Variables}
None
\subsubsection{Assumptions}
None
\subsubsection{Access Program Semantics}

\noindent init():
\begin{itemize}
    \item Transition: initialise pygame
\end{itemize}

\noindent mouse():
\begin{itemize}
    \item Transition: handle the mouse control response of the game
\end{itemize}

\noindent loadBackground():
\begin{itemize}
    \item Transition: load the game background
\end{itemize}

\noindent game():
\begin{itemize}
    \item Transition: main loop of the game
\end{itemize}



\section{MIS of Dragon Module}
\subsection{Interface Syntax}
\subsubsection{Exported Access Programs}
\begin{table}[h!]
    \centering
    \begin{tabular}{|p{4cm}|p{2cm}|p{2cm}|p{2cm}|}
    \hline
    \textbf {Name}  & {In} & {Out} & {Exceptions} \\
    \hline
    setSize & - & - & -\\
    \hline 
    \end{tabular}
\end{table}
\subsection{Interface Semantics}
\subsubsection{State Variables}
Not Applicable 
\subsubsection{Environmental Variables}
None
\subsubsection{Assumptions}
None
\subsubsection{Access Program Semantics}

\noindent setSize():
\begin{itemize}
    \item Transition: set the size of the dragon unit on the game board
\end{itemize}


\section{MIS of Enemy Module}
\subsection{Interface Syntax}
\subsubsection{Exported Access Programs}
\begin{table}[h!]
    \centering
    \begin{tabular}{|p{4cm}|p{2cm}|p{2cm}|p{2cm}|}
    \hline
    \textbf {Name}  & {In} & {Out} & {Exceptions} \\
    \hline
    setWave & - & - & -\\
    \hline
    setHP & - & float & -\\
    \hline
    setLevel & - & - & -\\
    \hline
    moveEnemy & - & - & -\\
    \hline
    drawEnemy & - & - & -\\
    \hline
    \end{tabular}
\end{table}
\subsection{Interface Semantics}
\subsubsection{State Variables}
Not Applicable 
\subsubsection{Environmental Variables}
None
\subsubsection{Assumptions}
None
\subsubsection{Access Program Semantics}

\noindent setWave():
\begin{itemize}
    \item Transition: spawn new enemy units for the current wave
\end{itemize}

\noindent setHP():
\begin{itemize}
    \item Transition: set the individual hit point (HP) of the enemy
    \item Output: the hit point (HP)
\end{itemize}

\noindent setLevel():
\begin{itemize}
    \item Transition: set the individual level of the enemy
\end{itemize}

\noindent moveEnemy():
\begin{itemize}
    \item Transition: increase the coordinate of the enemy by the movement speed
\end{itemize}

\noindent drawEnemy():
\begin{itemize}
    \item Transition: draw the enemy on the game board
\end{itemize}



\section{MIS of Bullet Module}
\subsection{Interface Syntax}
\subsubsection{Exported Access Programs}
\begin{table}[h!]
    \centering
    \begin{tabular}{|p{4cm}|p{2cm}|p{2cm}|p{2cm}|}
    \hline
    \textbf {Name}  & {In} & {Out} & {Exceptions} \\
    \hline
    setImage & - & - & -\\
    \hline
    getDirection & - & - & -\\
    \hline
    shotEnemy & - & boolean & -\\
    \hline
    moveEnemy & - & - & -\\
    \hline
    drawEnemy & - & - & -\\
    \hline
    \end{tabular}
\end{table}
\subsection{Interface Semantics}
\subsubsection{State Variables}
Not Applicable 
\subsubsection{Environmental Variables}
None
\subsubsection{Assumptions}
None
\subsubsection{Access Program Semantics}

\noindent setImage():
\begin{itemize}
    \item Transition: Transition: set the image of the bullet itself
\end{itemize}

\noindent getDirection():
\begin{itemize}
    \item Transition: set the direction of the bullet
\end{itemize}

\noindent shotEnemy():
\begin{itemize}
    \item Transition: determine if the enemy is within bound of the game
    \item Output: return ‘True’ if enemy is within bound of the game and hence can be damaged
\end{itemize}

\noindent moveBullet():
\begin{itemize}
    \item Transition: move the bullet towards the enemy
\end{itemize}

\noindent drawBullet():
\begin{itemize}
    \item Transition: draw the bullet on the screen
\end{itemize}

\section{MIS of Path Module}
\subsection{Interface Syntax}
\subsubsection{Exported Access Programs}
\begin{table}[h!]
    \centering
    \begin{tabular}{|p{4cm}|p{2cm}|p{2cm}|p{2cm}|}
    \hline
    \textbf {Name}  & {In} & {Out} & {Exceptions} \\
    \hline
    inPlay & int, int & boolean & -\\
    \hline
    onBoard & int, int & boolean & -\\
    \hline
    evolveBound & int, int & boolean & -\\
    \hline
    inParty & int, int & boolean & -\\
    \hline
    createPath & - & - & -\\
    \hline
    verticalPath & - & - & -\\
    \hline
    horizontalPath & - & - & -\\
    \hline
    \end{tabular}
\end{table}
\subsection{Interface Semantics}
\subsubsection{State Variables}
Not Applicable 
\subsubsection{Environmental Variables}
None
\subsubsection{Assumptions}
None
\subsubsection{Access Program Semantics}

\noindent inPlay():
\begin{itemize}
    \item input: the x and y coordinates
    \item Transition: determine if the object is inside the game screen
    \item Output: return true if the object is within the game screen
\end{itemize}

\noindent onBoard():
\begin{itemize}
    \item input: the x and y coordinates
    \item Transition: determine if the object is inside the game board area
    \item Output: return true if the object is within the game board area
\end{itemize}

\noindent evolveBound():
\begin{itemize}
    \item input: the x and y coordinates
    \item Transition: determine if the coordinates is within the evolve button
    \item Output: return true if the object is within the evolve button
\end{itemize}

\noindent inParty():
\begin{itemize}
    \item input: the x and y coordinates
    \item Transition: determine if the object is inside the party menu
    \item Output: return true if the object is within the party menu
\end{itemize}

\noindent createPath():
\begin{itemize}
    \item Transition: create the corners of the enemy path
\end{itemize}

\noindent verticalPath():
\begin{itemize}
    \item Transition: create the vertical portion of the enemy path
\end{itemize}

\noindent horizontalPath():
\begin{itemize}
    \item Transition: create the horizontal portion of the enemy path
\end{itemize}




\end{document}
