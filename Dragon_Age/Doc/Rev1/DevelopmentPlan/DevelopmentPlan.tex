\documentclass{article}
\usepackage[utf8]{inputenc}
\usepackage{color}}
\usepackage{soul}
\title{\textbf{Development Plan}}
\author{Group 8: Team Eight\\
        Stanley Liu\\
        Zhi Zhang\\
        Toni Miharja\\}
\date{December 6 2017}

\begin{document}

\maketitle
\newpage
\textcolor{red}{\tableofcontents}
\newpage

\textcolor{red}{\section{Revision History}}
\begin{table}[h!]
    \centering
    \textcolor{red}{
    \begin{tabular}{|p{2.5cm}|p{3cm}|p{3cm}|p{2cm}|}
    \hline
    \textbf {Date}  & {Developer} & {Change} & {Revision} \\
    \hline
    September 29, 2017 & Stanley Liu, Toni Miharja, Zhi Zhang & Development plan& Revision 0\\
    \hline
    December 6, 2017  & Zhi Zhang  & add table of contents, revision history table, project review.& Revision 1\\
    \hline
    \end{tabular}}
    \textcolor{red}{\caption{Revision History: Problem Statement}}
\end{table}

\newpage


\section{Team Meeting plan}
\subsection{When}
\begin{itemize}
    \item Regular meeting will be conducted on Mondays and Thursdays after lab as we will be attending labs and all members are free after the lab.
    \item Team members have agreed to keep the following timings free, reserved for the meeting:
    \begin{itemize}
        \item Monday, 6:30 - \textcolor{red}{8:30} pm
        \item Thursday, \textcolor{red}{4:30} - \textcolor{red}{6:30} pm
    \end{itemize}
    \item The duration of the meeting is flexible, depending on the need.
\end{itemize}

\subsection{Where}
Labs or school libraries will be the main location for meetings.

\subsection{Frequency}
The team will meet at \textcolor{red}{least} twice \st{once} a week. The team will meet with higher frequency depending on need and tasks required to be completed.

\subsection{Roles}
\begin{itemize}
    \item Team Leader: Toni
    \item Software Developers: Stanley, Zhi, Toni
    \item Software Tester: Toni
    \item Gitlab Manager: Stanley
    \item LaTex Editor: Zhi
\end{itemize}

\subsection{Rules For Meeting Agendas}
\begin{itemize}
    \item Meeting agenda will begin with topics that affect the entire team and continue with more specific details from each individual.
    \item The first topic will be to review the agenda to ensure that the team knows what to expect from the meeting.
    \item Meeting agenda will be based on input from team members.
    \item Agenda topics will be listed in questions.
    \item Each agenda item will have a realistic estimated time for discussion.
    \item One team member will be responsible to lead each topic.
    \item Meeting will end with a written statement on decisions made by the team.
\end{itemize}

\section{Team Communication Plan}
Our team will utilise a combination of communication channels to ensure effective communication:
\begin{itemize}
    \item WeChat group chat serves as the main communication channel to organise meetings and talk about the project in general.
    \item Physical meeting on a weekly basis ensures that each member knows what the other members are doing and ensure that everyone is on the same page to ensure timely project completion.
    \item Git issues is also utilised to keep track of the main issues in the project development.
\end{itemize}

\section{Team Member Roles}
\begin{itemize}
    \item Team leader: Toni
    \begin{itemize}
        \item The team leader sets the direction of the team, chairs meetings and ensures that the project is on track to completion.
    \end{itemize}
\end{itemize}

\begin{itemize}
    \item Software developers: Stanley, Zhi Zhang, Toni
    \begin{itemize}
        \item Software developer codes and develop the application.
    \end{itemize}
\end{itemize}

\begin{itemize}
    \item tester: Toni
    \begin{itemize}
        \item Software tester checks that the requirements of the project are met.
    \end{itemize}
\end{itemize}

\begin{itemize}
    \item Git Manager: Stanley
    \begin{itemize}
        \item Git Manager is in-charge of merging the different branches to master.
    \end{itemize}
\end{itemize}

\begin{itemize}
    \item LaTeX Editor: Zhi Zhang
    \begin{itemize}
        \item Latex Editor is in-charge of ensuring that the documentations of the project are formatted well in LaTeX.
    \end{itemize}
\end{itemize}

The members are aware that the roles can change and each member should be clear of which role he/she is taking.

\section{Git Workflow Plan}
We will adopt a feature-branch workflow plan. We will have a separate development (and the feature branches) and release branches.

\subsection{Label \textcolor{red}{Usage}}
Upon merging the different branches to master, we will tag the version of the product (for example ver 0.1, ver 1.2, etc).

\subsection{Milestones}
We have determined several key milestones in our project development and each milestone is broken down to several tasks. Currently we have planned for the following milestones (changes may be made along the way):
\begin{itemize}
    \item Requirement document
    \item POC demonstration
    \item Version 1
    \item Version 2
    \item Testing
    \item Final Version
\end{itemize}

\section{Proof of Concept Demonstration Plan}
For our team’s proof of concept demonstration, we would like to show the feasibility of the following:
\begin{itemize}
    \item Creating a working tower class, the attacking unit that users can put on the field,  that is capable of firing projectiles only within range.
    \item Creating a working enemy class, the unit that will walk through the designated path and have a hit point (HP).
    \item Ensuring a proper projectile collision detection between the projectile to the enemy.
    \item Creating a good layout design for the game.
\end{itemize}

\section{Technology}
The programming language we are using are Python 2.7 and pygame 1.9.3. For IDE we are using the Python default IDLE. Testing framework will be Python unittest. And we will use LaTex for document generation.

\section{Coding Style}
The team understands the importance of having a coding standard. We will be following the Google Python Style Guide to ensure that the code that the team generate is consistent throughout.

\section{Project Schedule}
Please refer to Tower-Defender/ProjectSchedule/projectSchedule.gan and Tower-Defender/Doc/Rev1/DevelopmentPlan//ProjectSchedule.pdf


\section{Project Review}
\textcolor{red}{The team communicated well developing the project and split work evenly between team members. Each team member completed his/her assigned work as expected, and learned the new language Pygame through the development. The design met all the functional and non-functional requirements of the original game, with addition of the team's own idea of game theme, game interface and optimization of game functionality such as adding three types of dragons that each has different effects on enemy. }
\end{document}

