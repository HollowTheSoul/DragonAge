
\documentclass[12pt]{article}
\usepackage{graphicx}
\usepackage{paralist}
\oddsidemargin 0mm
\evensidemargin 0mm
\textwidth 160mm
\textheight 200mm
\renewcommand\baselinestretch{1.0}
\pagestyle {plain}
\pagenumbering{arabic}
\newcounter{stepnum}
\title{Assignment 1}
\author{SFWR ENG 2AA4}
\date{Files due Jan 26, E-mail partner due Jan 27, Lab report due Feb 2}
\begin {document}
%\renewcommand{\labelenumi}{\alph{enumi}.}
\maketitle
The purpose of this software design exercise is to write a Python program that
creates, uses, and tests a simple Abstract Data Type (ADT) that stores data on a
circle.  The Circle ADT will allow a program to create instances of the datatype
{\tt CircleT}.  A circle ADT may be of interest in computer graphics or gaming
applications.  The program will consist of two modules and a test driver
program.
All of your code should be documented using doxygen.  All of your reports should
be written using LaTeX.
\section *{Step \refstepcounter{stepnum} \thestepnum}
Write a first module that creates a circle ADT.  It should consist of a Python code
file named {\tt CircleADT.py}.  The module should define a class CircleT, which
contains the following class methods that define the external interface:
\begin{itemize}
\item A constructor ({\tt CircleT}) that takes three real numbers $x$, $y$
  and $r$ as input and assigns them to private instance variables.  The $x$ and
  $y$ values define the centre of the circle and $r$ defines its radius.
\item Three getters named {\tt xcoord}, {\tt ycoord} and {\tt radius} that
  return the $x$ and $y$ coordinates of the centre of the circle and the radius
  of the circle, respectively.
\item A method named {\tt area} that returns the area of the circle.
\item A method named {\tt circumference} that returns the circumference of the circle.
\item A method named {\tt insideBox} that takes the following inputs: 
  the $x$ coordinate of the left side of a box
  ($x_0$), the $y$ coordinate of the top of a box ($y_0$), the width ($w$) of
  the box and the height ($h$) of the box.  The box, the circle and the
  associated coordinate system are shown in Figure~\ref{Fig_CircleBoxIntersect}.
  This method should return true if the circle is inside the box and false if
  it is not.
\begin{figure}
\begin{center}
%\rotatebox{-90}
{
 \includegraphics[width=0.6\textwidth]{CircleBoxIntersect.pdf}
}
\caption{\label{Fig_CircleBoxIntersect} Determination of whether a circle is
  inside a box or not}
\end{center}
\end{figure}
\item A method named {\tt intersect} that takes a second instance of {\tt
    CircleT} $c$ as input and returns true if the circles intersect and false
  otherwise.
\item A method named {\tt scale} that takes a float $k$ as an argument 
  and changes the radius such that it is scaled by $k$.
\item A method named {\tt translate} that take two floats $dx$ and $dy$ as
  arguments and translates the centre of the circle by $dx$ in the $x$ direction
  and by $dy$ in the $y$ direction.
\end{itemize}
\section *{Step \refstepcounter{stepnum} \thestepnum}
Write a second module that uses the first module to calculate various statistics
for a list of circles.  It should consist of the Python file: {\tt
  Statistics.py}.  Some of the routines in this module should be implemented
using the numpy, which is located at {\tt http://www.numpy.org/}.  The new
module should consist of the following functions:
\begin{itemize}
\item A function named {\tt average} that takes a list of instances of {\tt
    CircleT} and returns the average radius of all of the circles in the list.
  This function should be implemented using {\tt numpy}.
\item A function named {\tt stdDev} that takes a list of instances of {\tt
    CircleT} and returns the standard deviation of the radii of all of the
  circles in the list.  This function should be implemented using {\tt numpy}.
\item A function named {\tt rank} that takes a list of instances of {\tt
    CircleT} and returns a listed ranked by radius.  A ranking list provides for each
  element in the list the position it would hold if the list were sorted in
  descending order of radius.  The maximum entry in the list will have a rank of 1.  For
  instance, the rank of radii [6.0, 5.0, 11.0, 9.0] would be [3, 4, 1, 2].  You are
  required to implement this function yourself, without using {\tt numpy}.  The
  efficiency of your implementation is not relevant, only the correctness.  If
  you need to make assumptions to implement your algorithm, please state your
  assumptions as doxygen comments in the code.
\end{itemize}
\section *{Step \refstepcounter{stepnum} \thestepnum}
Write a third module that tests the first and second modules together.  It
should be a Python file named {\tt testCircles.py}.  Write a {\tt
  Makefile} with a rule {\tt make test} that runs your {\tt testCircles} source
with the Python interpreter.  Each procedure should have at least
one test case.  Record your rationale for test case selection and the results of
using this module to test the procedures in the other two modules.  The
requirements for testing are deliberately vague; at this time, we are most
interested in your ideas and intuition for how to build and execute your test
suite.
\section *{Step \refstepcounter{stepnum} \thestepnum}
Add to your makefile a rule for {\tt make doc}.  This rule should compile your
documentation into an html and LaTeX version.
\section *{Step \refstepcounter{stepnum} \thestepnum}
Submit the files {\tt CircleADT.py}, {\tt Statistics.py}, {\tt testCircles.py}
and {\tt Makefile} using git?.  E-mail the {\tt CircleADT.py} to your assigned
partner.  (Partner assignments will be posted.)  Your partner will
likewise e-mail you his or her files.  This step must be completed no later than
midnight of the deadline noted at the beginning of this assignment.
\section *{Step \refstepcounter{stepnum} \thestepnum}
After you have received your partner's files, replace your corresponding files
with your partner's.  Do not make any modifications to any of the code.  Run
your test module and record the results.  Your evaluation for this step does not
depend on the quality of your partner's code, but only on your discussion of the
testing results.
\section *{Step \refstepcounter{stepnum} \thestepnum}
Write a report (using LaTeX) that includes the following:
\begin{enumerate}
\item Your name and macid.
\item Your {\tt CircleADT.py}, {\tt Statistics.py}, {\tt testCircles.py} and {\tt Makefile} files.
\item Your partner's {\tt CircleADT.py} file.
\item The results of testing your files.
\item The results of testing your files combined with your partner's files.
\item A discussion of the test results and what you learned doing the exercise.
  List any problems you found with
\begin{inparaenum} 
\item your program,
\item your partner's module, and
\item the specification of the modules.
\end{inparaenum}
\item A discussion of how you handled the value of $\pi$ in your program and why
  you made this choice.  Is $\pi$ explicitly expanded in your formulae, or do
  you use a symbolic constant?  If you use a constant, what is its scope?
\end{enumerate}
Final submission over git? is due by midnight on the assigned due date.
\subsubsection*{Notes}
\begin{enumerate}
\item Please organize your git repository with the following directories at the
  top level: {\tt Assig1}, {\tt Assig2}, {\tt Assig3}, {\tt Assig4} and {\tt
    Assig5}. You are required to create all of these folders for the submission
  of Assignment 1.  The spelling and capitalization must be exactly as given
  above. ??
\item The files for this submission should be placed in the {\tt Assig1}
  folder. ??
\item Please put your name and macid at the top of each of your source
  files.
\item Your program must work in the ITB labs on moore when compiled using Python
  2.7.  
\item If your partner fails to provide you with a copy of his or her files by
  the deadline, please tell the instructor via e-mail as soon as possible.
\item If you do not send your files to your partner by the deadline, you will be
  assessed a \textbf{10\% penalty} to your assignment grade.
\item Any changes to the assignment specification will be announced in class.
  It is your responsibility to be aware of these changes.
\end{enumerate}
\end {document}



