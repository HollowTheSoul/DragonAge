\documentclass[12pt]{article}
\usepackage{graphicx}
\usepackage{paralist}
\oddsidemargin 0mm
\evensidemargin 0mm
\textwidth 160mm
\textheight 200mm
\renewcommand\baselinestretch{1.0}
\pagestyle {plain}
\pagenumbering{arabic}
\title{Tower Defender: Problem Statement}
\author{Group 8: Stanley Liu 001404020\\    
                 Toni Miharja\\
                 Zhi Zhang 400005778}
\begin {document}
%\renewcommand{\labelenumi}{\alph{enumi}.}
\maketitle

\section {What Problem are you trying to solve?}
As means of relaxation and refreshment, entertainment surely is an important part of many people’s lives. With the rise of technology, increasingly more people resort to video games on their laptop or phones as ways to pass time and entertain themselves. As developers and gamers ourselves, we aim to create a fun and intellectually stimulating game for the general public to enjoy. We are trying to reproduce a pokemon themed tower defense game on the pc platform with python and pygame.

\section {Why is this an important problem?}
Tower Defence is widely known to be intellectually challenging and fun by the gaming community. Mind games, like Tower Defence games, challenge the gamers’ brain, making them think , strategize and remember information. They help train the brain while making the process really fun for the user. We aim to develop  our Tower Defence game creatively and thoughtfully for the users.

The theme of the tower defense game is mostly known as the classic war theme, with placing defensive structure like soldier, cannon or fortress on player’s territories to obstruct enemy attackers. However, the theme of our game will be based on the different types of dragons. Each dragon will have its designed attribute, and will have advantage(does more damage than normal) over a certain type of dragon, at the same time a disadvantage over another type. We will set levels to dragons too as they do advanced attack form with leveling up. This design makes our game unique and more interesting.
\section {What is the context of the problem you are solving?}
    \subsection {Who are the stakeholders?}
    Multiple people will be related to the reproduction of this game. These people will include project leader, project developer, project tester, project advisors, and consumers. To be specific, the project leader, developer and tester are different roles assigned to team members. The project advisors are the teaching assistants and the professor for the course 2XA3. The consumers will be the users who play the game after the launch.

    \subsection {What is the environment for the software?}
    The original game is written in pygame and could run on any Windows, Mac and Linux machine. However, it requires the user to have python 2.7 and pygame installed. Therefore, we are going to convert the pygame package into an executable file for the ease of running.

\section{How are you going to solve the problem?}
First of all, every team member is going to learn pygame as we do not have prior knowledge of it. We would use the incremental and iterative software development method to break the original project into separate parts and have each developer working on it. For implementation, we will start with writing a simpler version of game, then develop it in depth. Meanwhile, we will ask customer feedback from our classmates in the course by having them play our game.

The process of the software development is the most important part we are focusing on, not just completing the code. We aim to take care of every detail and encourage plenty of team communication to make sure everything is clear along the way.


\end {document}



