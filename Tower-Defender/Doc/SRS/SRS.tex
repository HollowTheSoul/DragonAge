\documentclass{article}
\usepackage[utf8]{inputenc}
\usepackage{enumerate}


\title{Software Requirement Specification}
\author{Group 8: Team Eight \\
                 Stanley Liu (MacID: liuz23) \\    
                 Toni Miharja (MacID: miharjat)\\
                 Zhi Zhang (MacID: zhangz1)}
\date{October 5 2017 }
\usepackage[showgrame]{geometry}

\usepackage{titling}
\renewcommand\maketitlehooka{\null\mbox{}\vfill}
\renewcommand\maketitlehookd{\vfill\null}

\begin{document}
\maketitle
\newpage
\tableofcontents
\newpage
 
\section{Revision History}
\begin{table}[h!]
    \centering
    \begin{tabular}{|p{2.5cm}|p{3cm}|p{3cm}|p{2cm}|}
    \hline
    \textbf {Date}  & {Developer} & {Change} & {Revision} \\
    \hline
    October 4, 2017 & Stanley Liu & Added introduction. & Revision 0\\
    \hline
    October 5, 2017  & Stanley Liu, Toni Miharja, Zhi Zhang & Added functional requirement, non-functional requirement, Off-the-shelf solution. & Revision 0\\
    \hline
    \end{tabular}
    \caption{Revision History: Requirements Documentation}
\end{table}
 
\section{Introduction}

\subsection{Purpose}
The purpose of this requirement document is to give a detailed description for the tower defense game “Dragon Age”. This document will include the stakeholders and the goals for the project, and mainly specifying the functional and nonfunctional requirements. At the end, it contains a off-the-shelf solution related to our project.

\subsection{Scope}
“Dragon Age” is a tower defense game running on personal computer. Basically, the user will send out three types of dragons to locate alongside a designed route to defend the homeland. As the enemy wave increases, the dragons can be upgraded to do more damages.

\subsection{Stakeholders}
The stakeholders of the project are:
\begin{itemize}
    \item Software developer identifies our project group, which include project leader, project developer, and project tester. These people divides the project into different tasks to complete and make sure every task is done according to the project schedule.
    \item Project advisors include our professor and teaching assistants of the course, who give proper guidance and precious advices for the software development process.
    \item Ensuring a proper projectile collision detection between the projectile to the enemy.
    \item Consumers would be the users who play the game after the game launches, and we are targeting teenagers and adults specifically.
\end{itemize}

\subsection{Goals}
The project group has the following goals:
\begin{itemize}
    \item The main goal of the project is to focus on the details of the development process, making sure every step is completed with teamwork and follows the software principles.
    \item The project should have well-defined assignments for each group member. Every member should participate in team meetings and contribute evenly.
    \item We aim to meet all the requirement for the original game concept and add features for optimization. Although the theme is different, the requirement such as the rules of the game should be achieved.
    \item The software development should be incremental and iterative. GitHub is used as the depository and each member will contribute by merging from their individual branch to master branch.
    \item The final game needs to be produced to an executable file from python and pygame code.
\end{itemize}

\section{Functional Requirements}
Functional requirements are those requirements that provide the fundamental reason for the product's existence

\subsection{System Features}
\begin{itemize}
    \item The game shall provide mouse click function for players to play the game. % not a requirement, what do clicks DO
    \item The game shall start with providing three types of dragon (fire, water, and poison) and sufficient resource (money) for users to put at least one dragon on the map to defend the first wave of enemy. % utilize symbolic parameters
    \item The dragons shall have type attributes and each attacking form should have a specific effect on enemy. % what are the effects, damage level of attack?
    \item The users should earn resources after killing each enemy. The resource amount earned is based on the wave. % how to determine resources gained? formula?
    \item When users earn enough resources, they have either use them to upgrade dragons or buy new dragons. % use case, not a requirement
    \item The game shall have enemy waves that increase in movement speed and defense after each wave. % what does increase in defense mean?! More health, less health per attach by player?!
    \item If all the lives of our home base are taken by enemies, the game shall display the “game over” message. % do not use 'our', you are not a player
    \item The game shall enable the start and pause button for the ease of users to set up dragons in between waves. % split these req's into description and rationale
\end{itemize}

\newpage
\subsection{Use Cases}
\begin{table}[h!]
    \centering
    \begin{tabular}{|p{1.5cm}|p{1cm}|p{2cm}|p{2cm}|p{5cm}|}
    \hline
    \textbf{Name}  & {Actor} & {Precondition} & {Postcondition} & {Flow of events}\\
    \hline
    Build/buy dragon & User & The start up of the game & User is able to start defending the base using dragons. & 
    \begin{enumerate}[i]
        \item User checks if they have sufficient resources to buy a new dragon. 
        \item User selects the dragon to build (either fire, water or poison).
        \item User locates the block where they want to place the dragon.
    \end{enumerate}\\
    \hline
    Destroy an existing dragon & User & There is one or more existing dragons placed. & The selected dragon is demolished. & 
     \begin{enumerate}[i]
        \item User selects the dragon that he wishes to demolish.
        \item User click on the demolish button.
    \end{enumerate}\\
    \hline
     Upgrade an existing dragon & User & There is one or more existing dragons placed. & The selected dragon is upgraded. & 
    \begin{enumerate}[i]
        \item User checks if they have sufficient resources to upgrade a dragon
        \item User selects the dragon to upgrade.
        \item User click on the upgrade button.
    \end{enumerate}\\
    \hline
    \end{tabular}
    \caption{Use Cases}
\end{table}

\section{Non-Functional Requirements}
\subsection{Look and Feel Requirements}
\subsubsection{Appearance Requirements}
The layout of the game will induce the feel of Middle Earth. The background may vary between stages and all will revolve around the theme. The dragons will be different in colors according to their types. The program will be sized to fit properly with the size of the window. % not too bad, make sure to include fit criteria
\subsubsection{Style Requirement}
The game should bring an adventurous and mystical feel to the user. The gameplay will provide the user with a more intense feeling of concentration rather than laid back.

\subsection{Usability and Humanity Requirements}
\subsubsection{Ease of Use Requirements}
The game will be easily picked up by players of all ages. % even 6 months old?
\subsubsection{Personalisation and internationalisation requirements}
Not applicable $% change into a sentence
\subsubsection{Accessibility Requirements}
The game will be compatible with most operating systems. % which ones, be SPECIFIC

\subsection{Performance Requirements}
\subsubsection{Speed and latency requirements}
The game will be highly responsive, responding to the user input immediately. % how fast is immediately?
\subsubsection{Safety-Critical Requirements}
The game will not compromise the user device. % compromise in what way?
\subsubsection{Reliability and Availability Requirements}
The game will be available for use anywhere and anytime. It does not need internet connection and will be able to run smoothly as long as the user’s device system is not heavily loaded. % express in terms of percent uptime, however 100% is nearly impossible
\subsubsection{Scalability or Extensibility Requirements}
The code will allow for scalability and extensibility. % how, and when is this satisfied?

\subsection{Operational and Environmental Requirements}
\subsubsection{Expected Physical Environment}
The game will be available for use anywhere and anytime. It does not need internet connection and will be able to run smoothly as long as the user’s device system is not heavily loaded.
\subsubsection{Requirements for Interfacing with Adjacent Systems}
Not applicable.% change into a sentence
\subsubsection{Release Requirements}
The game will be updated yearly.

\subsection{Maintainability and Support Requirements}
\subsubsection{ Maintenance Requirements}
The team will make sure that the game will have minimum maintenance needs. % not a requirement
\subsubsection{Supportability Requirements}
As the game is compatible with most operating systems, most people with a running PC or laptop should be capable of running the game on their machines. % give me some minimum requirements!!!!
\subsubsection{Adaptability Requirements}
The game shall be universally accessible by users. % universally...? all languages?

\subsection{Security Requirements}
\subsubsection{Access Requirements}
The game will be accessible to those who are willing to play and have access to the necessary infrastructure. % what infra!!!
\subsubsection{Integrity Requirements}
The game will not alter its own source code and will not accept invalid user input.
\subsubsection{Privacy Requirements}
The game will not be able to access data outside of the game and hence will not infringe user’s privacy. % think more in terms of keeping user details private
\subsubsection{Audit Requirements}
Not applicable.% change into a sentence
\subsubsection{Immunity Requirements}
Not applicable.% change into a sentence

\subsection{Cultural Requirements}
The game will be available only in English. The game will not be offensive in any ways to any religious or ethnic groups.

\subsection{Legal Requirements}
\subsubsection{Compliance Requirements}
This game will not compromise any laws.

\section{Off-the-shelf Solutions}
When it comes to the existing solution, there is a already made production on GitHub named “Pokemon Tower Defense” which is the game we are going to redevelop in this course. The requirements we are using come from this existing game, however, we write our own game code and add new features.

 
\end{document}
